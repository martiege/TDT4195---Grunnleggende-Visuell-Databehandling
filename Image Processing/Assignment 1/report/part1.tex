\section{Spatial Filtering}

\subsection{Task 1: Theory}
\subsubsection*{a)}
Sampling is the process of converting a continuos-time signal to a discrete-time signal, usually by measuring the continuos-time signal at specific points in time and extending this measurement over a set time step. 

\subsubsection*{b)}
Quantization is the process of constraining a signal from a larger to a smaller set of values, like mapping colours to the standard RGB range of 256 integer values. 

\subsubsection*{c)}
A high contrast image histogram would look similar to a dirac delta function, with most values grouped together around the same intensity.

\subsubsection*{d)}
\begin{table}[]
    \label{tab:initial_image}
    f = \begin{tabular}{|l|l|l|l|l|}
        \hline
        5 & 0 & 2 & 3 & 4 \\ \hline
        3 & 2 & 0 & 5 & 6 \\ \hline
        4 & 6 & 1 & 1 & 4 \\ \hline
    \end{tabular}
\end{table}
\begin{align*}
    n_{\text{pixel}} &= 3 * 5 = 15 \\
    L = 7
    i_0 &= 2 \\ 
    i_1 &= 2 \\
    i_2 &= 2 \\
    i_3 &= 2 \\
    i_4 &= 3 \\
    i_5 &= 2 \\
    i_6 &= 2 \\
    i_7 &= 0
\end{align*}
Then using \cref{eq:equalizer} on \cref{tab:initial_image} gives \cref{tab:equalized_image}. 

\begin{align*}
    \begin{bmatrix}
        n   & 0             & 1             & 2             & 3             & 4             & 5             & 6             & 7 \\ \hline
        f_n &\frac{2}{15} &\frac{2}{15} &\frac{2}{15} &\frac{2}{15} &\frac{3}{15} &\frac{2}{15} &\frac{2}{15} &\frac{0}{15} \\
        F_n &\frac{2}{15} &\frac{4}{15} &\frac{6}{15} &\frac{8}{15} &\frac{11}{15} &\frac{13}{15} &\frac{15}{15} &\frac{15}{15}
    \end{bmatrix}
\end{align*}

\begin{equation}
    \label{eq:equalizer}
    g_{i,j} = floor((L - 1) * \sum_{n = 0}^{f_{i,j}} \frac{i_n}{n_{\text{pixel}}} )
\end{equation}

\begin{table}[]
    \label{tab:equalized_image}
    \begin{tabular}{|l|l|l|l|l|}
        \hline
        6 & 0 & 2 & 3 & 4 \\ \hline
        3 & 2 & 0 & 5 & 6 \\ \hline
        4 & 6 & 1 & 1 & 4 \\ \hline
    \end{tabular}
\end{table}
